\chapter{\abstractname}


This thesis explores the application of Large Language Models (LLMs) to enhance domain modeling in software architecture, specifically focusing on the extraction of bounded contexts as defined by Domain-Driven Design (DDD). Traditional domain modeling is a time-consuming and expertise-driven process, often leading to suboptimal architectures under tight deadlines. This research investigates how AI can assist in decomposing complex monolithic systems into modular, maintainable components.

A case study was conducted with FTAPI Software GmbH, a company facing the challenge of modernizing its legacy monolithic architecture. A five-phase, prompt-driven workflow was developed to guide LLMs through a structured analysis, including ubiquitous language extraction, event storming, bounded context identification, aggregate design, and technical architecture mapping. This methodology was applied to two distinct domains: SecuRooms, a well-defined and previously modularized system serving as a benchmark, and SecuMails, a complex monolithic system targeted for modernization.

The results show that LLMs can effectively generate viable bounded contexts and domain models that closely align with those created by experienced human architects, especially for domains with clear requirements like SecuRooms. For the more entangled SecuMails monolith, the LLM proposals provided a valuable starting point but struggled to capture implicit business rules and historical technical debt. Expert interviews confirmed the value of the LLM as an "architectural sparring partner" that accelerates initial design, enforces systematic analysis, and offers unbiased perspectives.

The thesis concludes that a semi-automated approach, combining the analytical speed of LLMs with the contextual judgment of human experts, offers a highly effective strategy for software architecture design. This collaborative model enables a more thorough exploration of architectural candidates, ultimately leading to more robust and maintainable systems.