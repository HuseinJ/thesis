\chapter{Related Work}\label{chapter:relatedwork}
This chapter reviews existing research relevant to this thesis and positions it within the broader academic and industrial context. By examining prior approaches, their findings, and limitations, the discussion highlights both the foundations this work builds upon and the gaps it seeks to address.

\section{Automated Domain Model Generation}\label{admg}
Domain modeling represents a time-intensive and expertise-dependent aspect of software engineering that requires deep understanding of both business requirements and technical constraints. In this process, engineers typically convert textual requirements into domain models that accurately represent the problem space and provide a foundation for solving business challenges. The inherent complexity and substantial resource demands of manual domain modeling have motivated researchers to explore automation approaches that could reduce both time investment and dependency on scarce domain expertise. Studies such as Chen et al. \autocite{chen2023automated} and from Saini et al. \autocite{Saini2022} explore the capabilities of Large Language Models to assist during this phase, while simultaneously highlighting the current limitations these models face in fully capturing domain semantics and business logic. 

\subsection{Fully Automated Domain Modeling Approaches}

Chen et al. \autocite{chen2023automated} conducted a comprehensive comparative study using GPT-3.5 and GPT-4 for fully automated domain modeling. Their findings reveal that while LLMs demonstrate impressive domain understanding capabilities, they remain impractical for full automation. Significantly, their research highlighted that LLM-generated domain models exhibit high precision but low recall, meaning that while the generated elements are often correct, many required domain elements are missing from the output. Furthermore, Chen et al. found that LLMs struggle most with identifying relationships between domain concepts compared to classes and attributes, and rarely incorporate established modeling best practices or complex design patterns.

\subsection{Semi-Automated Interactive Approaches}

In contrast to fully automated approaches, Saini et al. \autocite{Saini2022} propose a bot-assisted interactive approach that addresses the need for human expertise in domain modeling. Their work recognizes that domain modeling decisions require contextual knowledge and personal preferences that vary from engineers. Rather than attempting full automation, their approach generates multiple alternative solutions for domain modeling scenarios and learns from user preferences over time through an incremental learning strategy. Saini et al.'s work provides traceability between requirements and generated models, enabling users to understand and validate the AI's modeling decisions. This addresses a critical concern for enterprise adoption where architectural decisions must be explainable. Their approach specifically handles complex domain modeling patterns, which are also relevant to bounded context identification in Domain-Driven Design.

\subsection{Implications for Bounded Context Identification}

The findings from these studies have significant implications for bounded context identification in Domain-Driven Design. The research collectively demonstrates that while LLMs show promise in automated design generation, human expertise and interaction are essential for achieving practical, high-quality results in complex software architecture and modeling tasks. The semi-automatic approaches that combine AI assistance with human expertise appear more effective than fully automated solutions, particularly for enterprise environments where architectural decisions must be both accurate and explainable.

These insights support the approach taken in this thesis, which focuses on semi-automated bounded context identification that leverages LLM capabilities while maintaining human control and validation throughout the process. The evidence suggests that such hybrid approaches are more likely to succeed in real-world enterprise environments like FTAPI's modularization efforts, where domain expertise and contextual knowledge are critical for successful architectural transformations.

\section{Monolith Decomposition}

The evolution from monolithic to modular architectures represents one of the most significant paradigm shifts in software engineering over the past two decades. Understanding this evolution is crucial for appreciating both the challenges and opportunities in modern system decomposition approaches.

\subsection{The Evolution from Monolith to Modular Architectures}
Monolithic architectures emerged as the dominant pattern in enterprise software development during the 1990s and early 2000s. During this period, enterprise software was primarily deployed as single, large applications that contained all business logic, data access, and user interface components within a unified codebase. Fowler \autocite{fowler2002patterns} documented how enterprise applications naturally evolved into monolithic structures due to the technological constraints and development practices of this era. 

Over the years, numerous studies have highlighted the constraints and limitations of monolithic architectures as systems evolve and scale. Research has consistently demonstrated that as application size and complexity increase, significant architectural challenges emerge that impact both development efficiency and system maintainability.

Blinowski et al. \autocite{Blinowski2022} empirically demonstrated several critical bottlenecks inherent in monolithic systems. Their study revealed that as monolithic applications grow, "modifying the application's source becomes harder as more and more complex code starts to behave in unexpected ways." The research highlighted how architectural boundaries deteriorate over time, with developers finding it "increasingly harder to keep changes that related to a particular module to only affect this very module." This boundary erosion leads to a cascade effect where "changes in one module may lead to unexpected behavior in other modules and a cascade of errors."

The industry began exploring alternative approaches that could address the scalability, maintainability, and development velocity issues inherent in monolithic systems while preserving some of their operational simplicity. The emergence of modular architectures took several evolutionary paths. Richardson \autocite{richardson2019microservices} identified the modular monolith (modulith) as a pragmatic intermediate approach that maintains the deployment simplicity of monoliths while establishing clear module boundaries and enforcing architectural constraints. This approach allows organizations to achieve better separation of concerns and improved maintainability without the operational complexity of fully distributed systems.

The challenge of identifying optimal module boundaries has led to increased interest in Domain-Driven Design principles, particularly the concept of bounded contexts, as a systematic approach to decomposing monolithic systems. Evans' \autocite{evans2004domain} strategic design patterns provide a framework for identifying natural business boundaries that can serve as the foundation for modular architectures.


\section{Research Gap}
Despite the growing body of research in both automated domain modeling and monolith decomposition, there exists a notable gap at the intersection of AI-assisted architecture generation and Domain-Driven Design methodologies. While studies like Chen et al. \autocite{chen2023automated} and Saini et al. \autocite{Saini2022} have explored LLM capabilities for domain model generation, and numerous works have addressed monolith decomposition strategies, the specific application of AI assistance to DDD's bounded context identification remains largely unexplored.

Current research in automated domain modeling primarily focuses on generating UML diagrams or class structures from requirements, without explicitly considering the strategic patterns and bounded context principles fundamental to DDD. Similarly, existing monolith decomposition approaches often rely on technical metrics such as coupling and cohesion, or manual expert analysis, rather than leveraging the semantic understanding capabilities of modern LLMs to identify domain boundaries that align with business capabilities.

The research gap becomes particularly evident when considering that DDD provides a well-defined framework with clear architectural patterns and concepts—such as bounded contexts, aggregates, and ubiquitous language—that could serve as structured targets for AI-assisted generation. This structured nature of DDD makes it potentially more amenable to AI assistance than general domain modeling, yet this synergy remains underexploited in current literature.

This thesis addresses this research gap by proposing a semi-automated approach that specifically leverages LLMs to assist in bounded context identification within the DDD framework. By combining the semantic understanding capabilities of AI with the structured patterns of DDD, this work aims to provide a practical solution for organizations seeking to modernize their monolithic architectures while maintaining alignment between technical boundaries and business domains.